Planning, in general, consists of computing a sequence of actions which, starting in an initial state or states, will achieve a set of goal states. Classical planning, then, is a constrained planning task, which is fully observable, deterministic, finite, static (i.e. the world does not change unless the agent acts), and discrete (i.e. time, actions, objects and effects are not continuous). Nonclassical planning, by contrast, is used to describe planning in partially observable or stochastic environments. 

Typical problem-solving agents which make use of standard search algorithms (i.e. depth-first, breadth-first, $A^*$, etc) encounter several difficulties when solving classical planning problems; each of these must be kept in mind when designing a planning agent. First of all, the agent may be overwhelmed by ``irrelevant'' actions (that is, actions which do not necessarily bring the agent closer to achieving a goal). There are multiple ways within the classical planning community of discouraging a planning agent from considering irrelevant actions; regression search, for example, is one way to attempt to handle this problem. 